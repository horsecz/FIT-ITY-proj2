\documentclass[a4paper, twocolumn, 11pt]{article}
\usepackage[utf8]{inputenc}
\usepackage[IL2]{fontenc}
\usepackage[a4paper, text={18cm, 25cm}, left={1.5cm}, top={2.5cm}, twocolumn]{geometry}
\usepackage[czech]{babel}
\usepackage{times}
\usepackage{listings}
\usepackage{amsmath}
\usepackage{amssymb}
\usepackage{amsthm}
\usepackage{amsfonts}
\usepackage{mathtools}

\theoremstyle{definition}
\newtheorem{definition}{Definice}
\newtheorem{sentence}{Věta}

\date{}

\begin{document}

\begin{titlepage}

\begin{center}
\LARGE
\textsc{\Huge Fakulta informačních technologií\\[0.3em]
Vysoké učení technické v Brně}\\
\vspace{\stretch{0.382}}
Typografie a publikování\,–\,2. projekt\\[0.1em]
Sazba dokumentů a matematických výrazů
\vspace{\stretch{0.618}}
\end{center}
{\Large 2020 \hfill Dominik Horký (xhorky32)}

\end{titlepage}

\section*{Úvod}

V této úloze si vyzkoušíme sazbu titulní strany, ma\-te\-ma\-tic\-kých vzorců, prostředí a dalších textových struktur obvyklých pro technicky zaměřené texty (například rovnice~\eqref{eq:eq_2}
nebo Definice \ref{def:def_2} na straně \pageref{def:def_2}).\ Pro vytvoření těchto odkazů
používáme příkazy \verb|\label|, \verb|\ref| a \verb|\pageref|.

Na titulní straně je využito sázení nadpisu podle op\-ti\-cké\-ho středu s využitím zlatého řezu. Tento postup byl
probírán na přednášce.\ Dále je použito odřádkování se
zadanou relativní velikostí 0.4em a 0.3em.

\section{Matematický text}

Nejprve se podíváme na sázení matematických symbolů a~výrazů v plynulém textu včetně sazby definic a vět s využitím balíku \verb|amsthm|. Rovněž použijeme poznámku pod
čarou s použitím příkazu \verb|\footnote|. Někdy je vhodné
použít konstrukci \verb|${}$| nebo \verb|\mbox{}| která říká, že
(matematický) text nemá být zalomen. V následující definici je nastavena mezera mezi jednotlivými položkami
\verb|\item| na 0.05em.\setlength{\parskip}{1em}

\noindent
\begin{definition}\label{def:def_1}
Turingův stroj \textit{(TS) je definován jako šestice
tvaru} $M = (Q, \Sigma, \Gamma, \delta, q_0, q_F )$, \textit{kde:} \noindent
\end{definition} \setlength{\parskip}{0em}
\begin{itemize}
\setlength{\parskip}{0.07em}
\item $Q$ \textit{je konečná množina} vnitřních (řídících) stavů.\setlength{\parskip}{0.05em}
\item $\Sigma$ \textit{je konečná množina symbolů nazývaná} vstupní abeceda, $\Delta \notin \Sigma$,
\item $\Gamma$ \textit{je konečná množina symbolů}, $\Sigma \subset \Gamma$, $\Delta \in \Gamma$\textit{, nazývaná} pásková abeceda,
\item $\delta$\;:\;$(Q\setminus\{q_F\}) \times \Gamma \rightarrow Q \times (\Gamma \cup \{L,R\})$, \textit{kde} $L, R \notin \Gamma$\textit{, je parciální} přechodová funkce, \textit{a}
\item $q_0 \in Q$ \textit{je} počáteční stav \textit{a} $q_f \in Q$ \textit{je} koncový stav.
\end{itemize}\setlength{\parskip}{0.1em}

Symbol $\Delta$ značí tzv. \textit{blank} (prázdný symbol), který se
vyskytuje na místech pásky, která nebyla ještě použita.\setlength{\parindent}{1em}

\textit{Konfigurace pásky} se skládá z nekonečného řetězce,
který reprezentuje obsah pásky a pozice hlavy na tomto
řetězci. Jedná se o prvek množiny $\{ \gamma \Delta^\omega \mid \gamma \in \Gamma^* \} \times \mathbb{N}$\footnote{Pro libovolnou abecedu $\Sigma$ je $\Sigma^\omega$ množina všech \textit{nekonečných} řetězců nad $\Sigma$, tj. nekonečných posloupností symbolů ze $\Sigma$.}.
\textit{Konfiguraci pásky} obvykle zapisujeme jako $\Delta xyz\underline{z}x \Delta...$
(podtržení značí pozici hlavy). \textit{Konfigurace stroje} je pak
dána stavem řízení a konfigurací pásky. Formálně se jedná
o prvek množiny $Q \times \{ \gamma \Delta^\omega \mid \gamma \in \Gamma^* \} \times \mathbb{N}$.

\subsection{Podsekce obsahující větu a odkaz}
\setlength{\parskip}{0em}
\begin{definition} \label{def:def_2}
Řetězec $w$ nad abecedou $\Sigma$ je přijat TS~\textit{$M$ jestliže\- $M$ při aktivaci z počáteční konfigurace pásky}
$\underline{\Delta}w\Delta...$ \textit{a počátečního stavu} $q_0$ \textit{zastaví přechodem do koncového stavu} $q_F$\textit{, tj.} $(q_0, \Delta w \Delta^\omega, 0) \underset{M}{\overset{*}{\vdash}} (q_F,\gamma,n)$ \textit{pro nějaké} $\gamma \in \Gamma^*$ \textit{a} $n \in \mathbb{N}$.\par \setlength{\parindent}{1em}
\textit{Množinu} $L(M)$\ $=$\ $\{w\:|\:w$ \textit{je přijat TS $M$}$\} \subseteq \Sigma^*$ \textit{nazýváme} jazyk přijímaný TS $M$.\par
\end{definition}
\setlength{\parindent}{1em}
Nyní si vyzkoušíme sazbu vět a důkazů opět s použitím balíku \verb|amsthm|. \setlength{\parskip}{0em}
\begin{sentence}
\textit{Třída jazyků, které jsou přijímány TS, odpovídá} rekurzivně vyčíslitelným jazykům.
\end{sentence} \setlength{\parskip}{0em}
\begin{proof}
V důkaze vyjdeme z Definice \ref{def:def_1} a \ref{def:def_2}.
\end{proof}

\section{Rovnice}

Složitější matematické formulace sázíme mimo plynulý
text. Lze umístit několik výrazů na jeden řádek, ale pak je
třeba tyto vhodně oddělit, například příkazem \verb|\quad|.

$$\sqrt[i]{x_i^3} \quad\text{kde}\ x_i\ \text{je}\ i\text{-té sudé číslo}\quad y_i^{2 \cdot y_i} \neq y_i^{y_i^{y_i}}$$

V rovnici \eqref{eq:eq_1} jsou využity tři typy závorek s různou explicitně definovanou velikostí.
\setlength{\parskip}{0em}

\begin{eqnarray}
x & = & \left\{ \Big( \big[ a + b \big] * c\Big)^d \oplus 1 \right\} \label{eq:eq_1} \\ 
y & = & \underset{x \to \infty}{\lim}\;\frac{\sin^2{x} + \cos^2{x}}{\frac{1}{\log_{10} x}} \label{eq:eq_2}
\end{eqnarray}
\par \setlength{\parskip}{0.05em}
V této větě vidíme, jak vypadá implicitní vysázení limity $\lim_{n \to \infty} f(n)$ v normálním odstavci textu. Podobně je to i s dalšími symboly jako $\sum_{i=1}^n 2^i$\:či\:$\bigcap_{A \in B} A$. V případě vzorců $\lim\limits_{n \to \infty} f(n)$ a $\sum\limits_{i=1}^n 2^i$ jsme si vynutili méně úspornou sazbu příkazem \verb|\limits|.

\section{Matice}

Pro sázení matic se velmi často používá prostředí \verb|array| a závorky (\verb|\left|, \verb|\right|). \setlength{\parskip}{0.1em}
$$
\left(
\begin{array}{ccc}
    a + b & \widehat{\xi + \omega} & \hat{\pi}  \\
    \Vec{\mathbf{a}} & \overleftrightarrow{AC} & \beta
\end{array}
\right)
= 1\hspace{-1mm}\iff\\ \mathbb{Q} = \mathcal{R} \nonumber
$$
\par \setlength{\parindent}{0em} \setlength{\parskip}{-0.5em}
Prostředí \verb|array| lze úspěšně využít i jinde.

\setlength{\parskip}{1em}
$$
\binom{n}{k}
=
\left\{
\begin{array}{cl}
    0 & \text{pro}\ k\ < 0\ \text{nebo}\ k\ > n \\[0.05em]
    \frac{n!}{k!(n-k)!} & \text{pro}\ 0 \leq k\ \leq n.
\end{array}
\right. \nonumber
$$

\end{document}
